%\begin{abstract}
%\end{abstract}

\begin{center}
\bfseries
% Abstract in English
{\selectlanguage{english}Abstract}
\normalfont
\end{center}
\sloppy
In the manufacturing of printed circuit boards (PCBs), inspection of the solder joint ensures the quality and reliability of the product. Traditionally it is done by manual examination or by training a supervised machine learning model where an expert is used to label the dataset manually. This dependency on labeled data forms a bottleneck in efficiency and hence scaling in production. This means that with the increasing complexity of PCB designs and pressure to shorten production cycles, innovative inspection methods need to be developed for the detection of solder-joint defects without using an extensive prelabeled dataset.
Our research addresses this challenge by using unsupervised learning techniques for solder joint defect detection. We have leveraged the Anomalib library, a comprehensive collection of a wide range of anomaly detection algorithms for the implementation and evaluation of multiple models like PatchCore and DFM, among others. Our approach involves training or extracting features using these models from an unlabeled image dataset of solder joints. The models then label the solder joints as normal or abnormal, along with a corresponding prediction score for each image. This therefore removes the need to have a manually labeled training dataset and can turn out to be more scalable and adaptable for inspection requirements on PCBs. The PatchCore model provided very promising results, showing an accuracy of 91.25\%.
During the course of the thesis, we encountered many challenges that provided us with a lot of information regarding the practical application of these techniques. The most prominent ones are related to large training times for some models, which may hinder their feasibility in real-time manufacturing environments. [Other challenges will be added here later]. Looking towards future developments, a more extensive use of Vision Transformers will bring huge potential to this application. Although some ViT models are explored in our current work to some extent, there is much space for a more elaborate exploration of their potential in unsupervised defect detection. Further efforts could thus be oriented toward the development of such models, optimized with respect to faster training and inferencing times, and investigating their generalization capability for different solder joint defects.